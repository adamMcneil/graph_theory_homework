\documentclass{article}
\usepackage[pdf]{graphviz}
\usepackage{graphviz}
\usepackage{amsmath}


\begin{document}

CS5200 Homework 3 Graphs\\
Adam McNeil\\
Question 1) \\

Initialize n sets where n is the number of nodes and each set contains one of the nodes.
Then for each edge in the graph there are two possibilities for the connected nodes. \\
1) The nodes are in different sets\\
Then union the two set together and remove the odd sets\\
2) The nodes are in the same set\\
Then there is a cycle in the graph\\
If you run out of edges without finding a cycle there is no cycle in the graph
\digraph[scale=0.5]{questionone}{
    node [shape=circle]
    edge [arrowhead=none]
    subgraph {
        1 -> 2 [color = red]
        1 -> 3
        1 -> 5
        2 -> 4
        3 -> 6
        4 -> 5
    }
}
\{1, 2\} \{3\} \{4\} \{5\} \{6\} \\
\digraph[scale=0.5]{questiononeparttwo}{
    node [shape=circle]
    edge [arrowhead=none]
    subgraph {
        1 -> 2 [color = red]
        1 -> 3 [color = red]
        1 -> 5
        2 -> 4
        3 -> 6
        4 -> 5
    }
}
\{1, 2, 3\} \{4\} \{5\} \{6\} \\

\digraph[scale=0.5]{questiononepartthree}{
    node [shape=circle]
    edge [arrowhead=none]
    subgraph {
        1 -> 2 [color = red]
        1 -> 3 [color = red]
        1 -> 5 [color = red]
        2 -> 4
        3 -> 6
        4 -> 5
    }
}
\{1, 2, 3, 5\} \{4\} \{6\} \\

\digraph[scale=0.5]{questiononepartfour}{
    node [shape=circle]
    edge [arrowhead=none]
    subgraph {
        1 -> 2 [color = red]
        1 -> 3 [color = red]
        1 -> 5 [color = red]
        2 -> 4 [color = red]
        3 -> 6 
        4 -> 5
    }
}
\{1, 2, 3, 4, 5\} \{6\} \\

\digraph[scale=0.5]{questiononepartfive}{
    node [shape=circle]
    edge [arrowhead=none]
    subgraph {
        1 -> 2 [color = red]
        1 -> 3 [color = red]
        1 -> 5 [color = red]
        2 -> 4 [color = red]
        3 -> 6 [color = red]
        4 -> 5
    }
}
\{1, 2, 3, 4, 5\} \{6\} \\

\digraph[scale=0.5]{questiononepartfive}{
    node [shape=circle]
    edge [arrowhead=none]
    subgraph {
        1 -> 2 [color = red]
        1 -> 3 [color = red]
        1 -> 5 [color = red]
        2 -> 4 [color = red]
        3 -> 6 [color = red]
        4 -> 5
    }
}
\{1, 2, 3, 4, 5, 6\} \\

\digraph[scale=0.5]{questiononepartsix}{
    node [shape=circle]
    edge [arrowhead=none]
    subgraph {
        1 -> 2 [color = red]
        1 -> 3 [color = red]
        1 -> 5 [color = red]
        2 -> 4 [color = red]
        3 -> 6 [color = red]
        4 -> 5 [color = red]
    }
}
\{1, 2, 3, 4, 5, 6\} \\
Since 4 and 5 are already in the same set there is a cycle in the graph. \\
\\
2)\\
\digraph[scale=0.35]{questiontwo}{
    node [shape=circle]
    1 [label= "1 1|8"]
    2 [label= "2 2|7"]
    3 [label= "3 9|10"]
    4 [label= "4 3|6"]
    5 [label= "5 11|14"]
    6 [label= "6 12|13"]
    7 [label= "7 4|5"]
    1 -> 2 [label="T"]
    1 -> 3 [label="T"]
    1 -> 4 [label="F"]
    1 -> 7 [label="F"]

    2 -> 4 [label="T"]
    2 -> 7 [label="F"]

    3 -> 1 [label="B"]
    3 -> 7 [label="C"]

    4 -> 7 [label="T"]

    5 -> 6 [label="T"]
    5 -> 7 [label="C"]
 
    6 -> 4 [label="C"]
    6 -> 5 [label="B"]
}\\
3)\\
% \digraph[scale=0.5]{questionthree}{
%     node [shape=circle]
%     edge [arrowhead=none]
%     a -> b [label="4"]
%     a -> c [label="8"]
%     b -> c [label="9"]
%     b -> d [label="8"]
%     b -> e [label="10"]
%     b -> i [label="2"]
%     c -> d [label="2"]
%     c -> f [label="1"]
%     d -> e [label="7"]
%     d -> f [label="9"]
%     e -> f [label="5"]
%     e -> g [label="6"]
%     e -> h [label="6"]
%     e -> i [label="9"]
%     f -> g [label="2"]
%     g -> h [label="8"]
%     h -> i [label="5"]
% }\\
\digraph[scale=0.5]{Kruskal}{
    node [shape=circle]
    edge [arrowhead=none]
    a -> b [label="4"] [color = red]
    a -> c [label="8"]
    b -> c [label="9"]
    b -> d [label="8"]
    b -> e [label="10"]
    b -> i [label="2"] [color = red]
    c -> d [label="2"] [color = red]
    c -> f [label="1"] [color = red]
    d -> e [label="7"]
    d -> f [label="9"]
    e -> f [label="5"] [color = red]
    e -> g [label="6"]
    e -> h [label="6"] [color = red]
    e -> i [label="9"]
    f -> g [label="2"] [color = red]
    g -> h [label="8"]
    h -> i [label="5"] [color = red]
    }\\
Kruskal's algorithm \\
Join order: c-f, c-d, b-i, f-g, a-b, e-f, h-i, e-h\\
Prim's algorithm\\
Join order: a-b, b-i, i-h, h-e, e-f, c-f, c-d, f-g\\ 
4)\\
\digraph[scale=0.5]{maze}{
    node [shape=circle]
    edge [arrowhead=none]
    start -> i1
    i1 -> d1
    i1 -> i2
    i2 -> i3
    i2 -> i4
    i3 -> i5
    i3 -> i6
    i4 -> i7
    i4 -> i8
    i5 -> d2
    i6 -> i11
    i6 -> i13
    i7 -> i9
    i7 -> i11
    i8 -> d3
    i9 -> i10
    i10 -> d4
    i11 -> i12
    i12 -> i17
    i12 -> i18
    i13 -> i14
    i14 -> i15
    i14 -> i16
    i15 -> d5
    i16 -> d6
    i16 -> i17
    i17 -> d7
    i18 -> d8
    i18 -> finish
}\\
A DFS would be better in this case because we are not looking for the shortest path but only a
path. The DFS would return the first path that it found even if it was not the shortest path, 
but the BFS would be guaranteed to find the shortest path. \\

Bonus:\\
A bipartite graph cannot have a cycle with an odd number of edges.
This is equivalent to saying the nodes of the graph can be colored with two colors with no connected nodes being colored the same color. 
The following graph is a bipartite graph.
\digraph[scale=0.5]{bonuspartone}{
    node [shape=circle]
    edge [arrowhead=none]
    1[color=red]
    2[color=blue]
    3[color=blue]

    4[color=red]
    5[color=red]
    6[color=red]
    7[color=red]
    8[color=blue]
    9[color=blue]
    10[color=blue]
    11[color=red]
    1 -> 2
    1 -> 3
    2 -> 4
    2 -> 5
    3 -> 6
    3 -> 7
    4 -> 8
    5 -> 8
    6 -> 8
    7 -> 8
    7 -> 10
    7 -> 9
    9 -> 11
    }\\

\digraph[scale=0.5]{bonusparttwo}{
    node [shape=circle]
    edge [arrowhead=none]
    subgraph cluster_1 {
    
    1[color=red]
    4[color=red]
    5[color=red]
    6[color=red]
    7[color=red]
    11[color=red]
    }
    subgraph cluster_2 {
    2[color=blue]
    3[color=blue]


    8[color=blue]
    9[color=blue]
    10[color=blue]
    }
    1 -> 2
    1 -> 3
    2 -> 4
    2 -> 5
    3 -> 6
    3 -> 7
    4 -> 8
    5 -> 8
    6 -> 8
    7 -> 8
    7 -> 10
    7 -> 9
    9 -> 11
    }\\

Bonus 2: \\
Finding the square of a directed graph is equivalent to squaring the adjacency matrix as a matrix in math.
This is done though a series of column and row multiplications.  
The running time of a matrix multiplication is O(n^3), so that is how long it long take to calculate G^2.

\end{document}